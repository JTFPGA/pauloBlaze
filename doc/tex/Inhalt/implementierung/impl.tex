\chapter{Implementation}
\label{ch:impl}

The PicoBlaze is implemented by the direct description of LUTs.
This provides the desired speed and area advantages, but it is at the same time the biggest disadvantage.
The core only works with Xilinx FPGAs providing 6-input LUTs, limiting it to Spartan-6, Virtex-6 and all 7-Series devices.
\\
This chapter describes the implementation of the PauloBlaze.
It is written in pure VHDL code to provide maximum flexibility and portability.
\\
\\
\emph{
	This chapter aims to provide only an overview of the modules used in the PauloBlaze.
	Because it is fully compatible, one may refer to the original documentation \cite{KCPSM6} for specific details and timing diagrams. 
}

%\section{Overview}
%\label{sec:overview}
%The PauloBlaze takes always two clock cycles for one instruction.
%After the first cycle the address output is updated to request the next instruction from the program memory.

\section{Decoder}
\label{sec:decoder}
The decoder is a critical component in the processor.
It takes the complete 18-bit instruction word and evaluates it.
The analysis triggers the responsible component, for jumps and calls the program counter (\ref{sec:pc}), the register file (\ref{sec:reg}) for interaction with the Scratch Pad Memory, simple calculation and data manipulation operations are handled by the ALU (\ref{sec:alu}) and the I/O module (\ref{sec:io}) takes care of input and output operations.
\\ 
Depending on the type of instruction, different word structures are possible.
The most common instruction, after the 6-bit opcode, uses 4 bits to address the register and the remaining 8 bits as an immediate or to address a second register.

\begin{figure}[h]
	\centering
	\vspace{10px}
	\begin{bytefield}[endianness=big]{18}
		\bitheader{0, 7, 8 ,11, 12, 17} \\
		\bitbox{6}{opcode} & \bitbox{4}{reg A} & \bitbox{8}{immediate}
	\end{bytefield}
	\hspace{20px}
	\begin{bytefield}[endianness=big]{18}
		\bitheader{0, 3, 4, 7, 8 ,11, 12, 17} \\
		\bitbox{6}{opcode} & \bitbox{4}{reg A} & \bitbox{4}{reg B} & \bitbox{4}{unused}
	\end{bytefield}
	\caption{Bit patterns of common instruction}
\end{figure}

It is also possible to use those 8 bits as a parameter.
This is done in the shift and rotate operations extensively where the lower 4 bits determine which direction to shift and what happens to the shifted bit.
\\
The following opcodes are not used: $\text{17}_{16}$, $\text{23}_{16}$, $\text{27}_{16}$, $\text{2A}_{16}$, $\text{33}_{16}$, $\text{3B}_{16}$, $\text{3F}_{16}$
\\
External control signals like \emph{reset}, \emph{interrupt} or \emph{sleep} as well as internal ones (\emph{reset request}, \emph{zero}, \emph{carry}) will be processed by the decoder.


\begin{figure}[H]
	\sffamily
\label{fig:controlflow}
\centering
\includegraphics[width=1.0\textwidth]{controlflow.pdf}
\caption{Control Flow Graph}
\end{figure}

\section{Program Counter}
\label{sec:pc}
The program counter calculates the next address which is incremented by one every second cycle during normal execution.
The decoder drives the \emph{jmp\_addr} input used to determine the next address in case of a jump or a call.
If the latter occurs, the program counter stores the subsequent address onto a stack and jumps to the target.
After the interrupt or call has finished (using one of the return instructions), the topmost stack value is used as a jump target.
In case of an underflow, more returns than calls, or an overflow, more calls than entries on the stack, the program counter requests a reset of the whole processor.
It needs to be pointed out that an interrupt also needs one entry.
The size of the stack is a design parameter with the default value 30, but it can be as low as one, thus, only one interrupt or call may happen at the same time.

\section{Register File}
\label{sec:reg}
A PauloBlaze can access up to 32 8-bit registers.
They are split into two banks to be addressable by a 4-bit halfword.
The current bank can be set with the \emph{REGBANK} operation.
Only the ALU and the I/O module are capable of altering the registers.
The decoder controls the multiplexer selecting whose write enable signal and data is active.
The star operation is a special case in which the register file itself changes the content.
\\
To store larger amounts of data, this module contains the Scratchpad Memory (SPM).
With the generic \emph{scratch\_pad\_memory\_size} its size can be changed to 64, 128 or 256 bytes.
A bigger SPM demands more resources on the device.
\emph{STORE} and \emph{FETCH} operations are needed to transfer data between the SPM and the current register bank.

\begin{figure}[h]
	\sffamily
\label{fig:dataflow}
\centering
\includegraphics[width=1.0\textwidth]{dataflow.pdf}
\caption{Data Flow Graph}
\end{figure}

\section{ALU}
\label{sec:alu}
The module's primary purpose is to handle arithmetic, logical, shift and rotate operations.
They can alter the \emph{zero} and \emph{carry} flags which are handled in the ALU.
In case of an interrupt they will be backed up and restored after the interrupt has been handled.
\\
The first clock cycle is used to calculate the results, the second one to write them into the register file and to update the flags.
The PauloBlaze is a two address processor.
Thus, the ALU writes the result back into the first operands register, overriding it.

\section{I/O Module}
\label{sec:io}
The I/O module connects the PauloBlaze with the outside world.
It can read and write 8-bit words and address up to 256 channels with an 8-bit \emph{port\_id}.
Even though the user has two clock cycles to present or read the data, using a register before and after the ports is good practice and helps to meet the performance requirements.
Various high active strobe signals are asserted during the second cycle of a read or a write \hl{reading or a writing??} operation.
