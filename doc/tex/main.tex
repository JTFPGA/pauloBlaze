\documentclass[11pt, a4paper, twoside, cleardoubleempty, openright, pointlessnumbers]{scrreprt} %draft
%\usepackage[english]{babel}
\usepackage{setspace}
\usepackage{amsmath} 
\usepackage[utf8]{inputenc}							% UTF-8 tex file input incoding
\usepackage{amsmath}					% change font to be not so serific
\usepackage[T1]{fontenc}	
\usepackage{geometry}
\usepackage{subfigure}
\usepackage{listings}
\usepackage{algorithmic}
\usepackage{placeins}
\usepackage{threeparttable}
\usepackage[toc,page]{appendix}
\usepackage{soul}
\usepackage{siunitx}

\usepackage[automark,headsepline]{scrpage2}
\pagestyle{scrheadings}

\clearscrheadfoot
\automark[section]{chapter}
\lehead[]{\leftmark}
\rohead[]{\rightmark}
\ofoot[\pagemark]{\pagemark}

\usepackage{comment}
\usepackage{bytefield}
\usepackage{listings}
\usepackage{color}
\usepackage{graphicx}
\usepackage{float}
\usepackage[right]{eurosym}%Fuer Euro-Symbol
\geometry{left=25mm,right=25mm,top=30mm,bottom=3cm}
\setlength{\parindent}{0pt}
\parskip 6pt
\usepackage{lscape}
\usepackage[footnote, printonlyused]{acronym}
\usepackage[labelfont=bf]{caption}
%\captionstyle{\centerlastline}
\usepackage{hyperref}
\usepackage{booktabs}

\usepackage{listings}
\usepackage{tudscrcolor}
\usepackage{courier}

\usepackage[backend=biber,style=alphabetic]{biblatex}
\addbibresource{Inhalt/Literatur.bib}

\hypersetup{
%    bookmarks=true,         			% show bookmarks bar?t p
    unicode=false,          			% non-Latin characters in Acrobat?s bookmarks
    pdftoolbar=true,        			% show Acrobat?s toolbar?
    pdfmenubar=true,        			% show Acrobat?s menu?
    pdffitwindow=false,     			% window fit to page when opened
    pdfstartview={FitH},    			% fits the width of the page to the window
    pdftitle={PauloBlaze},    % title
    pdfauthor={Paul Genssler},     	% author
    pdfsubject={Documentation of the PauloBlaze},   				% subject of the document
    pdfcreator={Paul Genssler},   		% creator of the document
    pdfproducer={}, 	% producer of the document
    pdfkeywords={PauloBlaze, PicoBlaze, KCPSM6, Processor}, 				% list of keywords
    pdfnewwindow=true,     			% links in new window
    colorlinks=true,       			% false: boxed links; true: colored links
    linkcolor=black,          			% color of internal links
    citecolor=black,        			% color of links to bibliography
    filecolor=black,      				% color of file links
    urlcolor=black,           			% color of external links
    pdfstartpage=1
}

%------------------Anfang Nummerierung Anhang-----------------
 \renewcommand\appendix{\par 
   \setcounter{section}{0}% 
   \setcounter{subsection}{0}% 
   \setcounter{figure}{0}%s
   \renewcommand\thesection{\Alph{section}}% 
   \renewcommand\thefigure{\Alph{section}.\arabic{figure}}} 
%------------------Ende Nummerierung Anhang-----------------

\newcommand{\getTitle}{PauloBlaze}

\begin{document}
		
	\begin{center}
	\thispagestyle{empty}
	\mbox{}
	\vspace{3\baselineskip}
	
	\textsc{\Large Technische Universität Dresden\\
		Department of Computer Science\\
		Institute for Computer Engineering\\
		Chair for VLSI design, diagnostic and architecture\\ \vspace{2\baselineskip} }
		
	\vspace{2\baselineskip}
		
	\textbf{\LARGE Complex Training Period Processor Design \vspace{\baselineskip}}
	
	\vspace{2\baselineskip}
	
	\Large \getTitle{}

	\vspace{2\baselineskip}
		
	\large Paul Richard Genßler\\
	\small born on May 26, 1990 in Berlin\\
	\small (Mat.-Nr.: 3569856)\\
	\vspace{2\baselineskip}


	\end{center}

	\vspace{6\baselineskip}
	\begin{minipage}[t]{8cm}
		{\small Supervising Professor:}\\\large Prof. Dr.-Ing. habil. Rainer G. Spallek\\
	\end{minipage}
	
	\begin{minipage}[t]{8cm}
	{\small Adivsor:}\\\large Dipl.-Inf. Oliver Knodel\\
	\end{minipage}

	\vspace{2\baselineskip}


	Dresden, \today
	
	
	\newpage
	
	\setstretch{1,5}
	
	%---Leerseite--------->
	\newpage
	\mbox{}
	\thispagestyle{empty}
	\newpage
	%--------------------->
	
	
\setcounter{page}{3}

\setstretch{1,2}	

%\include{Inhalt/Kurzfassung}

%\include{Inhalt/Abstract}

	\newpage
	\markboth{}{}
	\setcounter{tocdepth}{3} %gliederung inhaltsverzeichnis
	\setcounter{secnumdepth}{3} %nummerierung tiefe
	\setstretch{1,3}% damit es genau auf die Seite passt
	 \tableofcontents

	\markboth{}{}
	\newpage
	
	\setstretch{1,2}
	
	\clearscrheadfoot
	\automark[section]{chapter}
	\lehead[]{\leftmark}
	\rohead[]{\rightmark}
	\ofoot[\pagemark]{\pagemark}
	
	\cleardoublepage		%???
	%---------------------------------------------------------------------------------------------------------------------------------------------------------------------
	\pagenumbering{arabic}
	\setcounter{page}{5}%Am Ende noch mal Kontrollieren!!!!
	\setstretch{1,2}

	\graphicspath{ {../diagrams/} }	
	
	\chapter{Introduction}
\label{ch:intro}

Field Programmable Gate Arrays (FPGAs) have always had a great potential for an efficient implementation of parallelizable algorithms.
Streaming based problems benefit from the low I/O latency and, thanks to the configurable cells, can be mapped very well.
However, step by step sequences have proven to be difficult. 
Complex tasks may demand a hybrid solution, configurable logic combined with a dedicated CPU on a single chip like a Xilinx Zynq or an Altera Cyclone.
In contrast state machines are suitable for short sequences and are often used.
But those state machines have practical limits, there is a huge bulk of states, one cannot cope with the number of transitions, more input values, more output signals, a more and more complex implementation leads to more resource usage.
This problem is not new and the typical solution is a so-called softcore.
Such a softcore contains the mapping of a CPU model onto the internal FPGA resources and the user can execute ordinary Assembler or C source code.
A popular core is the KCPSM6 also known as PicoBlaze \cite{PicoBlaze}. It is used in many applications like elliptic curve cryptography \cite{ref_elliptic}, a floating-point controller \cite{kadlec2005floating} or a multiprocessor system \cite{ref_paral_exec}.
Its implementation is utterly compact and with up to \SI{238}{\mega\hertz} very fast, however, because of its direct description of Look-Up Tables (LUTs) it is limited to current Xilinx FPGAs and not easily customizable.
\\ 
In this work a processor, called \textit{PauloBlaze}, is developed which is \SI{100}{\percent} compatible to the PicoBlazes ISA and all of the signal timings.
It should be very easy to replace a PicoBlaze in a current project, to modify this new implementation, or to deploy it without being restricted to particular platforms.
These benefits should outweigh the speed and area losses.
\\
This work is structured into four chapters.
An introduction and motivation is given in chapter~\ref{ch:intro} followed by chapter~\ref{ch:impl}, an overview on the implementation of the single modules and how they work together.
The resource usage is summarized in chapter~\ref{ch:res} and compared to the PicoBlazes.
Advantages and compromises of the PauloBlaze are discussed in the last chapter.

	\chapter{Implementation}
\label{ch:impl}

The PicoBlaze is implemented by the direct description of LUTs.
This provides the desired speed and area advantages, but it is at the same time the biggest disadvantage.
The core only works with Xilinx FPGAs providing 6-input LUTs, limiting it to Virtex-5, Spartan-6, Virtex-6 and all 7-Series devices.
\\
This chapter describes the implementation of the PauloBlaze, which is written in pure VHDL code to provide maximum flexibility and portability.
It consists of several components like the decoder (section~\ref{sec:decoder}), the program counter for jumps and calls (section~\ref{sec:pc}), the register file (section~\ref{sec:reg}) for interaction with the Scratch Pad Memory, simple calculation and data manipulation operations are handled by the ALU (section~\ref{sec:alu}) and the I/O module (section~\ref{sec:io}) takes care of input and output operations.
Figure \ref{fig:dataflow} presents these components and the data flow between them.
Because it is fully compatible to the PicoBlaze, one may refer to the original documentation \cite{KCPSM6} for specific details, available instructions and timing diagrams.

%\section{Overview}
%\label{sec:overview}
%The PauloBlaze takes always two clock cycles for one instruction.
%After the first cycle the address output is updated to request the next instruction from the program memory.
\section{Decoder}
\label{sec:decoder}
This critical component in the processor is the interconnection point in the design as it can be seen in figure \ref{fig:controlflow}.
It takes the complete 18-bit instruction word and evaluates it.
Depending on the type of instruction, different word structures are possible.
The most common instructions are displayed in figure \ref{fig:bit_word}, after the 6-bit opcode, it uses 4 bits to address the register and the remaining 8 bits as an immediate or to address a second register.

\begin{figure}[h]
	\centering
	\vspace{10px}
	\begin{bytefield}[endianness=big]{18}
		\bitheader{0, 7, 8 ,11, 12, 17} \\
		\bitbox{6}{opcode} & \bitbox{4}{reg A} & \bitbox{8}{immediate}
	\end{bytefield}
	\hspace{20px}
	\begin{bytefield}[endianness=big]{18}
		\bitheader{0, 3, 4, 7, 8 ,11, 12, 17} \\
		\bitbox{6}{opcode} & \bitbox{4}{reg A} & \bitbox{4}{reg B} & \bitbox{4}{unused}
	\end{bytefield}
	\caption{Bit Patterns of common Instruction}
	\label{fig:bit_word}
\end{figure}

It is also possible to use those 8 bits as a parameter.
This is done in the shift and rotate operations extensively where the lower 4 bits determine which direction to shift and what happens to the shifted bit.
\\
Based on the PicoBlaze instruction table \cite[p.\,54]{KCPSM6}, the opcodes $\text{17}_{16}$, $\text{23}_{16}$, $\text{27}_{16}$, $\text{2A}_{16}$, $\text{33}_{16}$, $\text{3B}_{16}$, $\text{3F}_{16}$ are free and can be used to implement new instructions.
\\
External control signals like \emph{reset}, \emph{interrupt} or \emph{sleep} as well as internal ones (\emph{reset request}, \emph{zero}, \emph{carry}) will be processed by the decoder.

\begin{figure}[H]
	\sffamily
	\centering
	\includegraphics[width=1.0\textwidth]{controlflow.pdf}
	\caption{Control Flow Graph}
	\label{fig:controlflow}
\end{figure}

\section{Program Counter}
\label{sec:pc}
The next address is calculated by the program counter, which increments it by one every second cycle during normal execution.
The decoder drives the \emph{jmp\_addr} input used to determine the next address in case of a jump or a call.
If the latter occurs, the program counter stores the subsequent address onto a stack and jumps to the target.
After the interrupt or call has finished (using one of the return instructions), the topmost stack value is used as a jump target.
In case of an underflow, more returns than calls, or an overflow, more calls than entries on the stack, the program counter requests a reset of the whole processor.
It needs to be pointed out that an interrupt also needs one entry.
The size of the stack is a design parameter with the default value 30, but it can be as low as one, thus, only one interrupt or call may happen at the same time.

\section{Register File}
\label{sec:reg}
A PauloBlaze can access up to 32 8-bit registers.
They are split into two banks to be addressable by a 4-bit halfword.
The current bank can be set with the \emph{REGBANK} operation.
Only the ALU and the I/O module are capable of altering the registers.
The decoder controls the multiplexer selecting whose write enable signal and data is active.
The star operation is a special case in which the register file itself changes the content.
\\
To store larger amounts of data, this module contains the Scratchpad Memory (SPM).
With the generic \emph{scratch\_pad\_memory\_size} its size can be changed to 64, 128 or 256 bytes.
A bigger SPM demands more resources on the device.
\emph{STORE} and \emph{FETCH} operations are needed to transfer data between the SPM and the current register bank.

\begin{figure}[h]
	\sffamily
\centering
\includegraphics[width=1.0\textwidth]{dataflow.pdf}
\caption{Data Flow Graph}
\label{fig:dataflow}
\end{figure}

\section{ALU}
\label{sec:alu}
The module's primary purpose is to handle arithmetic, logical, shift and rotate operations.
They can alter the \emph{zero} and \emph{carry} flags which are handled in the ALU.
In case of an interrupt they will be backed up and restored after the interrupt has been handled.
\\
The first clock cycle is used to calculate the results, the second one to write them into the register file and to update the flags.
The PauloBlaze is a two address processor.
Thus, the ALU writes the result back into the first operands register, overriding it.

\section{I/O Module}
\label{sec:io}
A connection to outside world is established by the I/O module.
It can read and write 8-bit words and address up to 256 channels with an 8-bit \emph{port\_id}.
Even though the user has two clock cycles to present or read the data, using a register before and after the ports is good practice and helps to meet the performance requirements.
Various high active strobe signals are asserted during the second cycle of a read or a write operation.

	\chapter{Evaluation}
\label{ch:res}
%transition
Implementing a design is only one step in the development, other important steps are verification and evaluation of the result.
%issue
%claim: schlechter aber konkurenzfähig
%dev s: veri dann eval
Even before the evaluation it was clear that a certain performance loss has to be accepted when writing pure VHDL instead of directly describing vendor specific elements, however the PauloBlaze still delivers good performance combined with an acceptable area increase.
%limit
%support
%nicht optimiert da anpassbarkeit wichtiger
Those measured values are based on a solution without any significant optimizations to keep the code readable, portable and easily changeable.
%conclusion
After all these are the project's main goals and that required a trade-off between them or area and speed.


\section{Verification}
\label{sec:verifi}
The whole PauloBlaze was simulated and later used in bigger designs to detect possible bugs.
To automate the simulation, a self testing program (cf.~\hyperref[{ch:test_prog}]{Appendix~\ref*{ch:test_prog}}) issues every data chancing instruction and checks the result afterwards.
Those checks helped to test jumps and other control flow instructions.
After the core passed the simulations, it was deployed into the PicoBlaze-Library \cite{picoLib} and performed well on Xilinx and Altera chips.
On the basis of the simulation and the usage in a bigger design, it can be said that there are no known bugs.

\section{Timing}
\label{sec:timing}
Several tests were conducted to show the good performance regarding one of the PicoBlazes key aspects: \hl{S/s}peed.
The first test scenario is a simple I/O design based the PicoBlaze manual \cite[p.\,72]{KCPSM6} where one can also find the values used in this comparison.
The frequency of the PauloBlaze was increased step by step until the Xilinx tool \emph{trace} \hl{(? link)} resulted in a timing violation.

\begin{table}[h]
	\sffamily
	\centering
	\captionof{table}{Achievable Speeds}
	\sisetup{detect-family=true}
	\label{tab:timings}
	\begin{tabular}{lS[table-format=1]|S[table-format=3,table-space-text-post = \si{\mega\hertz}]|S[table-format=3,table-space-text-post = \si{\mega\hertz}]|S[table-format=2.1,table-space-text-post = \si{\percent}]}
		Family		&	{Speed Grade}	&	{PicoBlaze}	&	{PauloBlaze}&	{Slowdown}	\\	\hline
		Kintex-7	&	-1			&	185\,\si{\mega\hertz}	&	156\,\si{\mega\hertz} 	&	15.7\,\si{\percent}		\\
					&	-3			&	238\,\si{\mega\hertz}	&	222\,\si{\mega\hertz} 		&	6.7\,\si{\percent}	\\	\hline
		Virtex-7	&	-3			&	232\,\si{\mega\hertz}	&	222\,\si{\mega\hertz} 		&	4.3\,\si{\percent}	\\	\hline
		Virtex-6	&	-3			&	238\,\si{\mega\hertz}	&	200\,\si{\mega\hertz} 		&	16.0\,\si{\percent}		\\	\hline
		Spartan-6	&	-1L			&	82 \,\si{\mega\hertz}	&	74\,\si{\mega\hertz} 		&	9.8\,\si{\percent}		\\		
					&	-3			&	136\,\si{\mega\hertz}	&	114\,\si{\mega\hertz} 		&	16.2\,\si{\percent}		\\
	\end{tabular}
\end{table}

It is not surprising to get the same speed from two different 7-Series devices because they share the same architecture.
Also, this architecture and the good speed grade fit the implementation best.
On the other hand, a Spartan-6 with a low grade less affected than one with a better grade.

The PicoBlaze-Library mentioned in section \ref{sec:verifi} was augmented and tested on the Atlys Board \cite{Atlys}, featuring a Spartan-6 (speed grad -3).
Both, PicoBlaze and PauloBlaze, were able to meet the timing requirements of \SI{100}{\mega\hertz} in this real world example.

\section{Resource Usage}
\label{sec:res_use}
Similar to the timing results the PauloBlaze utilizes more resources than its optimized, but restricted, counterpart as shown below with a few examples.
The first design was taken from the PicoBlaze Library and implemented on the same Atlys Board \cite{Atlys} \hl{nochmal linken?}.
It contains a module called SoFPGA featuring extensions like a divider unit for the single processor.
The \emph{Pico constrained} represents the normal PicoBlaze instantiation whereas all the placement instructions were deleted in the \emph{unconstrained} version.
Without the directives the tools could place parts of the processor into other modules.
Such modules may also occupy a part of a slice, but the usage report assigns the whole slice to the processor.
Thus, the resulting utilization values are only estimates.
All values were taken after the Xilinx tool \emph{map} had finished.

\begin{figure}[h]
	\sffamily
	\centering
	\includegraphics[width=1.0\textwidth]{res_sofpga}
	\caption{Resource Usage of the surrounding SoFPGA Module}
	\label{fig:res_sofpga}
\end{figure}

\begin{table}[h]
	\sffamily
	\centering
	\sisetup{detect-family=true}
	\label{tab:res_sofpga}
	\caption{Resource Usage of the surrounding SoFPGA Module}
	\begin{tabular}{l S[table-format=3] S[table-format=3] S[table-format=3] S[table-format=2.1,table-space-text-post = \si{\percent}] S[table-format=3.1,table-space-text-post = \si{\percent}]}
		\toprule
		Resource				&	{Pico unconstrained} & {Pico constrained} & {PauloBlaze} & \multicolumn{2}{c}{increase} \\ \midrule
		Slices			&	1266	&	1242	&	1348	& 6.5\,\si{\percent}	& 8.5\,\si{\percent}	\\	
		Slice Register	&	2924	&	2925	&	2928	& 0.1\,\si{\percent}	& 0.1\,\si{\percent}	\\	
		LUTs			&	2420	&	2402	&	2599	& 7.4\,\si{\percent}	& 8.2\,\si{\percent}	\\	
		LUTRAM			&	774		&	774		&	781		& 0.9\,\si{\percent}	& 0.9\,\si{\percent}	\\
		\bottomrule
	\end{tabular}
\end{table}

The differences between the two PicoBlaze versions are most likely caused by the placers behaviour to spread out and use more of the chip when there is enough space left. It is not surprising to see an additional demand of resources because of the PauloBlaze. \hl{vllt noch sagen, dass es nich soo viel ist?} However, the increase, especially to the constrained version, is higher than expected compared to the following measurements.

On basis of the same utilization report a direct look at the single processor is possible. It needs to be pointed out that the report is less precise because of the small number of resources used and the shifting of those between different components.

\begin{figure}[h]
	\sffamily
	\centering
	\includegraphics[width=1.0\textwidth]{res_mod1}
	\caption{Resource Usage of an embedded P*Blaze}
	\label{fig:res_mod}
\end{figure}

\begin{table}[H]
	\sffamily
	\centering
	\sisetup{detect-family=true}
	\caption{Resource Usage of an embedded P*Blaze}	
	\label{tab:res_mod}
	\begin{tabular}{l S[table-format=3] S[table-format=3] S[table-format=3] S[table-format=2.1,table-space-text-post = \si{\percent}] S[table-format=3.1,table-space-text-post = \si{\percent}]}
		\toprule
		Resource			&	{Pico unconstrained} & {Pico constrained} & {PauloBlaze} & \multicolumn{2}{c}{increase} \\ \midrule
		Slices			&	57	&	33	&	100	& 75.4\,\si{\percent}	& 203.0\,\si{\percent}	\\	
		Slice Register	&	79	&	79	&	83	& 5.0\,\si{\percent}	& 5.1\,\si{\percent}	\\	
		LUTs			&	108	&	129	&	155	& 43.5\,\si{\percent}	& 20.2\,\si{\percent}	\\	
		LUTRAM			&	49	&	49	&	56	& 14.3\,\si{\percent}	& 14.3\,\si{\percent}	\\
		\bottomrule
	\end{tabular}
\end{table}

The shifting of resources from one module to another becomes clear when comparing the slice and LUT numbers of the two PicoBlaze versions.
On one hand the unconstrained version shares "its" slices with other components almost doubling the usage, on the other hand \SI{17}{\percent} of the LUTs are not counted.
The PauloBlaze shows an additional slice usage by 43 or 67.
This is less than the 100 extra slices suggested by the report of the SoFPGA module.

To get a precise and reasonable result, the timing sections I/O design was tested with different chip families.
It is a simple design with only a few extra constructs around the processor.
The normal constrained PicoPlaze was used and compared to the PauloBlaze.

\begin{table}[h]
	\sffamily
	\centering
	\sisetup{detect-family=true}
	\caption{Resource Usage of a Simple I/O Design}
	\label{tab:area}
	\begin{tabular}{l c S[table-format=3]S[table-format=3] c S[table-format=3]S[table-format=3] c S[table-format=3.1,table-space-text-post = \si{\percent}]S[table-format=3.1,table-space-text-post = \si{\percent}]}
		\toprule			
		\textbf{Family}		&	&	\multicolumn{2}{c}{\textbf{PicoBlaze}}	&&	\multicolumn{2}{c}{\textbf{PauloBlaze}}	&&	\multicolumn{2}{c}{\textbf{increase}}	\\	
					&&	{S. LUTs}	& {S. Reg}	&& {S. LUTs}	& {S. Reg}	&&	{S. LUTs}	& {S. Reg}	\\	\midrule
		Spartan-6	&&	114		&	84		&&	275		& 90		&&	141.2\,\si{\percent}	& 7.1\,\si{\percent}		\\
		Virtex-6	&&	121		&	115		&&	276		& 91		&&	128.1\,\si{\percent}	& -20.9\,\si{\percent}		\\
		Virtex-7	&&	113		&	83		&&	283		& 82		&&	150.4\,\si{\percent}	& -1.2\,\si{\percent}		\\		\bottomrule
	\end{tabular}
\end{table}

Although this additions seem to be significant in percentage, it is a negligible increase regarding the overall utilization from \SI{1.7}{\percent} to \SI{3.1}{\percent} on a medium sized Spartan-6 (XC6SLX75).

	\chapter{Summary}
\label{ch:sum}
The primary goal of this work was to create a fully compatible processor to the PicoBlaze.
This has been achieved by developing the PauloBlaze, a processor that can replace the other and neither hardware nor software code have to be changed.
Even though there are some compromises, the PauloBlaze offers useful features enabling it to compete.
The PicoBlaze, a hand optimized design, is smaller than the new alternative.
The user has to compensate an increase of up to \SI{150}{\percent}, but because of the initially low requirements, the overall resource usage increases only marginally.
It is also faster than a PauloBlaze, it varies from a \SI{16.0}{\percent} to an only \SI{4.3}{\percent} slowdown.
If the design does not require maximum speed or if fast 7-Series FPGAs are used, the difference is negligible.
\\
On the other hand the new processor provides the possibility for specific optimization or an easy implementation of new instructions, which may save many clock cycles.
Furthermore the designer is not limited to the Spartan-6, Virtex-6 or 7-Series devices, the vendor independent description can be deployed everywhere, as long as the target supports VHDL.
\\
The PauloBlaze trades a small performance penalty for a far more adaptive model, providing flexible designers a very useful tool.
	
	%---Leerseite--------->
	\newpage
	\mbox{}
	\thispagestyle{empty}
	\newpage
	%--------------------->	
	
	\begin{appendices}
	\chapter{Test Program Listing}
\label{ch:test_prog}



%\definecolor{HKS44}{cmyk/RGB/rgb}{%
% 	1.00,0.50,0.00,0.00/156,177,219/0,0.34901960784,0.63921568627%
%}
%\definecolor{HKS57}{cmyk/RGB/rgb}{%
%   	1.00,0.00,0.90,0.20/000,122,071/0,0.47843137254,0.28235294117%
%}
%\definecolor{HKS33}{cmyk/RGB/rgb}{%
%  	0.50,1.00,0.00,0.00/129,026,120/0.50588235294,0.10196078431,0.47058823529%
%}

\lstdefinelanguage{psm}{% no default language
	keywords=[1]{% define all keywords
		LOAD	,
		ADD	,
		ADDCY	,
		AND	,
		CALL	,
		CALL@	,
		COMPARE	,
		COMPARECY	,
		DISABLE INTERRUPT	,
		ENABLE INTERRUPT	,
		FETCH	,
		HWBUILD	,
		INPUT	,
		JUMP	,
		JUMP@	,
		&,
		OR	,
		OUTPUT	,
		OUTPUTK	,
		REGBANK,
		RETURN	,
		RETURNI 	,
		RL	,
		RR	,
		SL0	,
		SL1	,
		SLA	,
		SLX	,
		SR0	,
		SR1	,
		SRA	,
		SRX	,
		STAR	,
		STORE	,
		SUB	,
		SUBCY	,
		TEST	,
		TESTCY	,
		XOR	,
		CONSTANT
	},
	morekeywords=[2]{%
		C, NC, Z, NZ %
	},
	morekeywords=[3]{%
		s0, s1, s2, s3, s4, s5, s6, s7, s8, s9, sA, sB, sC, sD, sE, sF%
	},
	deletekeywords={%
		0C%
	},
	sensitive=false,
	morecomment=[l]{;},
	morecomment=[s]{*}{*}
}

\lstset{%
	tabsize=4,
	numbers=left,
	numberstyle=\ttfamily,
	language=psm,
	basicstyle=\footnotesize\ttfamily		%\scriptsize,
	columns=fixed,
	showstringspaces=false,
	extendedchars=true,
	breaklines=true,
	showtabs=false,
	showspaces=false,
	showstringspaces=false,
	keywordstyle=[3]\bfseries\color{HKS36!80},
	keywordstyle=[2]\bfseries\color{HKS41!60},
	keywordstyle=[1]\bfseries\color{HKS44!60},
	commentstyle=\color{HKS57!90},
	stringstyle=\color[rgb]{0.627,0.126,0.941},%
	  moredelim=[is][keywordstyle]{|>}{<|},%
	  moredelim=[is][keywordstyle]{|<}{>|}%
}

\lstinputlisting{Inhalt/appendix/testcode.psm}
\end{appendices}
	
			%---Leerseite--------->
	\newpage
	\mbox{}
	\thispagestyle{empty}
	\newpage
	%--------------------->
	
	\setstretch{1,1}
	
%	\pagenumbering{Roman}
%	\setcounter{page}{104}
	\addcontentsline{toc}{chapter}{Bibliography}
	\printbibliography
		
\end{document}